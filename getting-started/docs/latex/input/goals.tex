
\section{Goals}
\label{sec:getting-started:getting-started:goals}

The main goals of this \SelfRef\ are:
\begin{itemize}
\item introduce the reader to basic modeling features such as: importing library modules, using a Haskell interpreter, using helper functions, composing functions, designing with layers, understanding type signatures, using basic input/output.
\item provide a step-by-step guide for modeling a toy system expressing concerns from four layers: function, extended behavior, model of computation and recursive/parallel composition. 
\item describe the above system as executing with the semantics dictated by four MoCs: synchronous dataflow (SDF), synchronous (SY), discrete event (DE) and continuous time (CT). For this purpose the system will be first instantiated multiple times using specialized helpers, and then described as a network of patterns overloaded with MoC semantics by injecting the right data types, thus exposing the polymorphism of atoms. 
\item briefly introduce the concepts of atoms and patterns and their usage and guide through creating custom patterns and behaviors.
\end{itemize}


%%% Local Variables:
%%% TeX-command-default: "Make"
%%% mode: latex
%%% TeX-master: "../report"
%%% End:
