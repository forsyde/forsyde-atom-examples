\documentclass{standalone}
\usepackage[american]{circuitikz}

\begin{document}
\begin{circuitikz}
  \draw
  (0,0) node[rground] {}
  to[V=VDD]      (0,2)
  to[spst=$\mathit{sw}_1$] (1.3,2) coordinate (r1)
  to[R=R]        (3,2) node[circ] (n1) {}
  to[C,l_=C]     (3,0) node[rground] (g1) {}
  (n1)
  to[ospst=$\mathit{sw}_2$](5,2) node (n2) {}
  to[R=R]        (5,0) node[rground] {}
  (n1) to      ++(0,1) node[ocirc] (n3) {}
  (n3) node[anchor=west] {$V_O$}
  ;
  % \draw[red,dashed] (r1)++(0,.7) rectangle ($(g1)+(.6,0)$);
  % \node[red,anchor=south east,xshift=-2mm] at (g1) {$f_{RC}$}; 
\end{circuitikz}
\end{document}

%%% Local Variables:
%%% mode: latex
%%% TeX-master: t
%%% End:
