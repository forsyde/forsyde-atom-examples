\documentclass[10pt]{article}
\usepackage[a4paper]{geometry}
\usepackage[footnotesize]{caption}
\usepackage{longtable}
\usepackage{booktabs}
\usepackage{hyperref}
\usepackage{enumitem}
\usepackage{marginnote}
\usepackage{listings}
\usepackage[tikz,math]{forsyde}

\title{The \ForSyDeLaTeX\ utilities}
\author{
  George Ungureanu \\
  Department of Electronic Systems\\
  KTH Royal Institute of Technology\\
  Stockholm, SWEDEN
}
\date{\today}



\newenvironment{optionslist}[0]{ 
\begin{list}{}{
	\setlength{\itemindent}{-10pt}
%	\setlength{\topsep}{0pt}
	\setlength{\itemsep}{0pt}
	\setlength{\parsep}{0pt}
}}{\end{list}}
\newcommand\bookmark[1]{\marginpar{\ttfamily #1}}
\lstset{
  basicstyle=\footnotesize\ttfamily,
  % numbers=left,
  frame=single,
  numberstyle=\tiny\color{black!30},
  commentstyle=\color{blue}\textit,
  stringstyle=\color{magenta}\textit,
  flexiblecolumns=false,
  basewidth={0.5em,0.45em},
  breaklines=true,
  language={[LaTeX]TeX},
  texcsstyle=*\color{red}\bfseries,
  keywordstyle=\color{blue}\bfseries,
  morekeywords={tikzpicture,document},
  moretexcs={trans,standard,interface,basic,cluster,node,path,embed},
}
\def\opt#1{\color{gray}{#1}}
\def\man#1{\color{black}{#1}}

\begin{document}
\maketitle
\reversemarginpar

\begin{abstract}
This is the reference manual for the \LaTeX\ utilities used in the context of \ForSyDe. All packages and their API features are documented here.
\end{abstract}

\section{Introduction}

This library was developed as an effort to standardize symbols and graphical primitives in documents related to \ForSyDe, but also to provide tools and utilities for user convenience. \ForSyDe is a high-level design methodology aiming at synthesizing correct-by-construction systems through formal means. For more information check \url{https://forsyde.ict.kth.se}.

The library contains the following main packages:
\begin{itemize}
\item \texttt{forsyde-tikz} : is a collection of \textsc{PGF} and \textsc{TikZ} styles, graphical primitives and commands for drawing \ForSyDe process networks;
\item \texttt{forsyde-math} : is a collection of math symbols used in the \ForSyDe formal notation. It is mainly focused on the ongoing \ForSyDeAtom methodology;
\item \texttt{forsyde-plot} : provides utilities for plotting \ForSyDe signals;
\item \texttt{forsyde-legacy} : API for the previous versions of this library.
\end{itemize}

\section{Installation \& usage}

There are three main alternatives to install the libraries:

\begin{enumerate}
\item copy the contents of \texttt{forsyde-latex/src} and \texttt{forsyde-latex/fonts} in their appropriate path under \texttt{TEXMFHOME} or any standard loading path, as specified by your \LaTeX\ compiler. Refer to \url{https://en.wikibooks.org/wiki/LaTeX/Installing_Extra_Packages} for more information.
\item compile your document with the variable \texttt{TEXINPUTS} set to \texttt{/path/to/forsyde-latex/src/}. If you intend to use \texttt{forsyde-math} characters, you need to generate the fonts under \texttt{forsyde-latex/fonts} using a \texttt{METAFONT} tool suite, and afterwards compile your document with the variable \texttt{TEXFONTS} set to \texttt{/path/to/forsyde-latex/fonts/};
\item copy the contents of \texttt{forsyde-latex/src} and \texttt{forsyde-latex/fonts} in the same folder as your document and compile normally.
\end{enumerate}

To include any of the packages enumerated in the introduction, you cal load the \texttt{forsyde} package with the appropriate option:

\begin{verbatim}
	\usepackage[option]{forsyde}
\end{verbatim}
where \texttt{option} is
\begin{itemize}
\item \texttt{tikz} for loading the \texttt{forsyde-tikz} library
\item \texttt{math} for loading the \texttt{forsyde-math} library
\item \texttt{plot} for loading the \texttt{forsyde-plot} library
\item \texttt{legacy} for loading the \texttt{forsyde-legacy} library
\end{itemize}

When loaded without an option, this package only provides some general commands for typesetting and logos:

\begin{longtable} { c | c }
  \toprule
  \textbf{Command}  & \textbf{Expands to} \\
  \midrule
  \texttt{\string\ForSyDe}      & \ForSyDe \\
  \texttt{\string\ForSyDeLaTeX} & \ForSyDeLaTeX \\
  \texttt{\string\ForSyDeAtom} & \ForSyDeAtom \\
  \bottomrule
\end{longtable}

\newpage
\tableofcontents
\newpage
% A library with graphical primitives for ForSyDe process networks
%
% Author: George Ungureanu, KTH - Royal Institute of Technology, Sweden
% Version: 0.3
% Date: 2015/05/20
\NeedsTeXFormat{LaTeX2e}
\RequirePackage{pgfplots}
\RequirePackage{pgfkeys}
\RequirePackage{xparse,l3regex}
\RequirePackage{ezkeys}
\RequirePackage{xstring}
\usetikzlibrary{decorations.markings, shapes, positioning, calc, fit, backgrounds, intersections, arrows}

\ProvidesPackage{forsyde-tikz}
              [2017/02/25 v0.4.0 ForSyDe TikZ Library]

\usetikzlibrary{forsyde.shapes,forsyde.paths,forsyde.nodes}
\pgfplotsset{compat=1.13}

%%%%%%%%%%%%%
% CONSTANTS %
%%%%%%%%%%%%%
% Colors
\newcommand{\defaultdrawcolor}{black}     		% draw color of signal paths
\newcommand{\defaultfillcolor}{white}     		% draw color of signal paths
\definecolor{sycolor}{RGB}{148,183,215}
\definecolor{ctcolor}{RGB}{225,119,19}
\definecolor{decolor}{RGB}{80,229,154}
\definecolor{sdfcolor}{RGB}{220,220,20}
\definecolor{blackboxcolor}{gray}{0.80}
% line widths of
\newlength{\sepq}\pgfmathsetlength{\sepq}{2pt}
\newcommand{\compositelinewidth}{.4pt}       % composite process line width
\newcommand{\skeletonlinewidth}{1pt}         % parallel processes line width
\newcommand{\signalpathlinewidth}{1pt}       % signal paths
\newcommand{\functionpathlinewidth}{.5pt}    % function paths
\newcommand{\vectorpathlinewidth}{3pt}       % vector paths
% sizes, etc.
\newcommand{\tokensize}{2pt}
\newcommand{\halftokensize}{1pt}

%%%%%%%%%%%%%%%%%%%%%%%%
% GENERIC TIKZ HELPERS %
%%%%%%%%%%%%%%%%%%%%%%%%
% Positioning of node text.
% #1 = node label
% #2 = label text
\newcommand{\textaboveof}[2]{\pgftext[bottom,at=\pgfpointanchor{#1}{north},y=+1mm]{#2}}%
\newcommand{\textrightof}[2]{\pgftext[left,  at=\pgfpointanchor{#1}{east}, x=+1mm]{#2}}%
\newcommand{\textbelowof}[2]{\pgftext[top,   at=\pgfpointanchor{#1}{south},y=-1mm]{#2}}%
\newcommand{\textleftof} [2]{\pgftext[right, at=\pgfpointanchor{#1}{west}, x=-1mm]{#2}}%


% Conditional if node was defined.
% #1 = node label
% #2 = true-statement
% #3 = false-statement
\long\def\ifnodedefined#1#2#3{%
    \@ifundefined{pgf@sh@ns@#1}{#3}{#2}%
}

\newcommand{\gettikzx}[2]{%
  \tikz@scan@one@point\pgfutil@firstofone#1\relax
  \edef#2{\the\pgf@x}%
}
% Get y-coordinate of node
\newcommand{\gettikzy}[2]{%
  \tikz@scan@one@point\pgfutil@firstofone#1\relax
  \edef#2{\the\pgf@y}%
}
% Get x- and y- coordinate of node
\newcommand{\gettikzxy}[3]{%
  \tikz@scan@one@point\pgfutil@firstofone#1\relax
  \edef#2{\the\pgf@x}%
  \edef#3{\the\pgf@y}%
}

% Decorate process ports with info
% #1 = node label
\newcounter{iportnum}
\newcounter{oportnum}
\newcommand\resetportinfo[1]{%
  \setcounter{iportnum}{0}
  \setcounter{oportnum}{0}
  \def\currentnode{#1}
}
\newcommand\wpinfo[2][south east]{%
  \addtocounter{iportnum}{1}
  \node[anchor=#1] at (\currentnode.w\theiportnum) {\tiny #2};
}
\newcommand\epinfo[2][south west]{%
  \addtocounter{oportnum}{1}
  \node[anchor=#1] at (\currentnode.e\theoportnum) {\tiny #2};
}


\tikzset{% 
    anch/.style={circle, draw=none, fill=red, inner sep=0pt, minimum size=3pt},
    label/.style={font=\ttfamily\scriptsize, text=red}
}




%%%%%%%%%%%%%%%%%%%%%%%%%%% 
% SHAPES OF MAIN ELEMENTS %
%%%%%%%%%%%%%%%%%%%%%%%%%%% 
\pgfkeys{/forsyde keys/.is family, /forsyde keys,
  primitive/.style  = {shape=atom shape},
  primitiven/.style = {shape=nary atom shape},
  process/.style    = {hasmoc, shape=leaf shape},
  processn/.style   = {hasmoc, shape=leafn shape},
  composite/.style  = {shape=comp shape},
  compositen/.style = {shape=nary comp shape},
  black box/.style  = {shape=leaf shape, moc=blackbox, inner xsep = 5pt,inner ysep = 5pt,},
  embed/.style      = {hasmoc, shape=leaf shape, inner sep=15pt},
  farmstyle/.style  = {shape = dp shape,inner xsep = 15pt,inner ysep = 20pt,},
  pipestyle/.style  = {shape = pipe shape,inner xsep = 15pt,inner ysep = 20pt,},
  skeleton/.style   = {shape = generic skel shape,inner xsep = 15pt,inner ysep = 20pt,},
  zipx/.style       = {transition={s1}{v1},rotate shape=180,type=zipx,},
  unzipx/.style     = {transition={s1}{v1},type=unzipx},
  transition/.code 2 args = {\edef\theshape{trans shape #1#2}\tikzset{/forsyde keys/shape = \theshape}},
  transition/.default     ={v1}{v1},
}
\newpage
\vfill
\begin{figure}[htb]\centering
  {%
    \setlength{\fboxsep}{7pt}%
    \setlength{\fboxrule}{1pt}%
    \fbox{\includegraphics[width=.9\textwidth]{figs/example-forsyde-math}}
  }
\end{figure}
\lstinputlisting{figs/example-forsyde-math.tex}
\vfill
\newpage

\section{The \texttt{forsyde-math} package}
\label{sec:forsyde-math-package}

This package provides a set of symbols and commands for writing equations mainly for the \ForSyDeAtom\ theoretical framework.

\subsection{Operator symbols}
\label{sec:fonts}

The \texttt{forsyde-math} package exports a set of symbols written in the \textsc{METAFONT} language (see \ref{sec:font-map}). These symbols are typed in using commands following the naming convention:

\begin{verbatim}
  \<name>               % inside a math environment
  \text<name>           % inside a text environment
\end{verbatim}

\noindent where the symbol name is from the table below\footnote{the symbol naming scheme reflects their semantics defined in the \ForSyDeAtom\ framework, as two acronyms: first one denoting the layer and the second one denoting the constructor}:\bookmark{operator symbols}

\def\makesymbolrow#1{{\tiny #1} & {\scriptsize #1} & {\footnotesize #1} & {\small #1} & {\normalsize #1} & {\large #1} & {\Large #1} & {\LARGE #1} & {\huge #1} & {\Huge #1}}
\begin{longtable} { c | c c c c c c c c c c }
  \toprule
  \texttt{<name>}  & \textbf{5pt} & \textbf{7pt} & \textbf{8pt} & \textbf{9pt} & \textbf{10pt} & \textbf{12pt} & \textbf{14.4pt} & \textbf{17.28pt} & \textbf{20.74pt} & \textbf{24.88pt} \\
  \midrule
  \texttt{BhFun} & \makesymbolrow{\textBhFun} \\
  \texttt{BhApp} & \makesymbolrow{\textBhApp} \\
  \texttt{BhDef} & \makesymbolrow{\textBhDef} \\
  \texttt{BhPhi} & \makesymbolrow{\textBhPhi} \\
  \texttt{BhDeg} & \makesymbolrow{\textBhDeg} \\
  \midrule
  \texttt{MocFun} & \makesymbolrow{\textMocFun} \\
  \texttt{MocApp} & \makesymbolrow{\textMocApp} \\
  \texttt{MocCmb} & \makesymbolrow{\textMocCmb} \\
  \texttt{MocPre} & \makesymbolrow{\textMocPre} \\
  \texttt{MocPhi} & \makesymbolrow{\textMocPhi} \\
  \texttt{MocDel} & \makesymbolrow{\textMocDel} \\
  \midrule
  \texttt{SkelFrm} & \makesymbolrow{\textSkelFrm} \\
  \texttt{SkelPip} & \makesymbolrow{\textSkelPip} \\
  \texttt{SkelFun} & \makesymbolrow{\textSkelFun} \\
  \texttt{SkelApp} & \makesymbolrow{\textSkelApp} \\
  \texttt{SkelRed} & \makesymbolrow{\textSkelRed} \\
  \texttt{SkelRec} & \makesymbolrow{\textSkelRec} \\
  \bottomrule
\end{longtable}

\subsection{Math commands}
\label{sec:miscellaneous}

There are a couple of macros defined for math environments, mainly for convenience. They are listed below:\bookmark{math commands}
{\footnotesize
\begin{longtable} {  p{2.5cm} | c | c | p{6cm}  }
  \toprule
  \textbf{Command} & \textbf{Example} & \textbf{Result} & \textbf{Explanation}  \\
  \midrule
  \texttt{\string\id\{name\}} & \texttt{\$\string\id\{func\}\$} & $\id{func}$
  & wraps \texttt{name} as an identifier, rather than loose characters \\
  \texttt{\string\context\{ctx\}\{f\}} & \texttt{\$\string\context\{\string\Gamma\}\{f\}\$} & $ \context{\Gamma}{f} $
  & associates context \texttt{ctx} to function \texttt{f} \\
  \texttt{\string\Constructor} \texttt{\{name\}\{layer\}} & \texttt{\$\string\Constructor\{all\}\{T\}\$} & $ \Constructor{all}{T} $
  & generic infix name for constructor in a user-defined layer \\
  \texttt{\string\BhCons\{name\}} & \texttt{\$\string\BhCons\{default\}\$} & $ \BhCons{default} $
  & infix name for constructor in the \emph{behavior~layer} \\
  \texttt{\string\MocCons\{name\}} & \texttt{\$\string\MocCons\{mealy\}\$} & $ \MocCons{mealy} $
  & infix name for constructor in the \emph{MoC~layer} \\
  \texttt{\string\SkelCons\{name\}} & \texttt{\$\string\SkelCons\{mesh\}\$} & $ \SkelCons{mesh} $
  & infix name for constructor in the \emph{skeleton~layer} \\
  \texttt{\string\SkelVec\{exp\}} & \texttt{\$\string\SkelVec\{v\}\$} & $ \SkelVec{v} $
  & surrounds \texttt{exp} in vector type delimiters, associated with the \emph{skeleton layer} \\
  \bottomrule
\end{longtable}
}

\subsection{Font map}
\label{sec:font-map}



The \ForSyDeAtom\ operators from \ref{sec:fonts} have been created using \textsc{METAFONT} and have been bundled as a font family called \texttt{forsydeatom}. These fonts can be imported in accordance to the \LaTeXe\ standard. The math symbol font based on this font family is called \texttt{atomoperators} and it declares all symbols as binary operators.

In case you need to access the fonts directly (and not through the \texttt{forsyde-math} package), here is the mapping of the \texttt{forsydeatom} font family:
 
\begin{figure}[h]\centering
  \includegraphics[clip, trim=2.2cm 21cm 2cm 4cm, width=\textwidth]{figs/testfont}
\end{figure}

%%% Local Variables:
%%% TeX-command-default: "Make"
%%% mode: latex
%%% TeX-master: "../refman"
%%% End:

% \newpage
% \begin{figure}[htb]\centering
\includegraphics[width=\textwidth]{figs/example-forsyde-plot}
\end{figure}
\lstinputlisting{figs/example-forsyde-plot.tex}
\newpage

\section{The \texttt{forsyde-plot} package}
\label{sec:forsyde-plot-package}


%%% Local Variables:
%%% TeX-command-default: "Make"
%%% mode: latex
%%% TeX-master: "../refman"
%%% End:

\newpage
\section{The \texttt{forsyde-legacy} package}
\label{sec:legacy-package}

This package offers an API for the legacy commands defined in older versions of the \ForSyDeLaTeX utilities. This way, documents compiled with old commands can be compiled with the newer versions of their respective library.

\subsection{\texttt{forsyde-tikz v0.3} or prior}
\label{sec:forsyde-tikz-v0.3}

Although from \texttt{v0.4} onward the draw commands have been heavily modified, the old commands could be mapped to the new API.

\begin{lstlisting}
\primitive[keys]         {id}{pos}{label}
\primitiven[keys]        {id}{pos}{label}
\leafstd[keys]           {id}{pos}{label}
\leafcustom[keys]        {id}{pos}
\compositestd[keys]      {id}{clustered nodes}{label}
\compositebbox[keys]     {id}{pos}{label}
\patterncluster[keys]    {id}{clustered nodes}{label}
\patternnodestd[keys]    {id}{pos}
\patternnodecustom[keys] {id}{pos}
\end{lstlisting}

\subsection{\texttt{forsyde-pc v0.3} or prior}
\label{sec:forsyde-pc-v0.3}

This package is obsolete and used to hold helpers associated to some \ForSyDe process constructors.

\begin{lstlisting}
\delay    [moc=,f1=,inner sep=,reverse]        {id}{pos}{label}
\delayn   [moc=,f1=,f2=,inner sep=,reverse]    {id}{pos}{label}
\map      [moc=,f1=,inner sep=,reverse]        {id}{pos}{label}
\comb     [moc=,f1=,inner sep=,reverse]        {id}{pos}{label}
\combII   [moc=,f1=,inner sep=,reverse]        {id}{pos}{label}
\combIII  [moc=,f1=,inner sep=,reverse]        {id}{pos}{label}
\combIV   [moc=,f1=,inner sep=,reverse]        {id}{pos}{label}
\scanl    [moc=,f1=,f2=,inner sep=,reverse]    {id}{pos}{label}
\scanlII  [moc=,f1=,f2=,inner sep=,reverse]    {id}{pos}{label}
\scanlIII [moc=,f1=,f2=,inner sep=,reverse]    {id}{pos}{label}
\scanld   [moc=,f1=,f2=,f3=,inner sep=,reverse]{id}{pos}{label}
\scanldII [moc=,f1=,f2=,f3=,inner sep=,reverse]{id}{pos}{label}
\scanldIII[moc=,f1=,f2=,f3=,inner sep=,reverse]{id}{pos}{label}
\moore    [moc=,f1=,f2=,f3=,inner sep=,reverse]{id}{pos}{label}
\mooreII  [moc=,f1=,f2=,f3=,inner sep=,reverse]{id}{pos}{label}
\mooreIII [moc=,f1=,f2=,f3=,inner sep=,reverse]{id}{pos}{label}
\mealy    [moc=,f1=,f2=,f3=,inner sep=,reverse]{id}{pos}{label}
\mealyII  [moc=,f1=,f2=,f3=,inner sep=,reverse]{id}{pos}{label}
\mealyIII [moc=,f1=,f2=,f3=,inner sep=,reverse]{id}{pos}{label}
\source   [moc=,f1=,f2=,inner sep=,reverse]    {id}{pos}{label}
\filter   [moc=,f1=,f2=,inner sep=,reverse]    {id}{pos}{label}
\hold     [moc=,f1=,inner sep=,reverse]        {id}{pos}{label}
\fillS    [moc=,f1=,f2=,inner sep=,reverse]    {id}{pos}{label}

\zip     [moc=,reverse]{id}{pos}
\zipIII  [moc=,reverse]{id}{pos}
\zipIV   [moc=,reverse]{id}{pos}
\zipV    [moc=,reverse]{id}{pos}
\zipVI   [moc=,reverse]{id}{pos}
\unzip   [moc=,reverse]{id}{pos}
\unzipIII[moc=,reverse]{id}{pos}
\unzipIV [moc=,reverse]{id}{pos}
\unzipV  [moc=,reverse]{id}{pos}
\unzipVI [moc=,reverse]{id}{pos}

\domaininterface[moc=,reverse]          {id}{pos}
\mocinterface   [mocin=,mocout=,reverse]{id}{pos}

\composite[ni=,no=,inner xsep=,inner ysep=,reverse] {id}{included}{label}
\blackbox [ni=,no=,inner xsep=,inner ysep=,reverse] {id}{included}{label}

\farm     [ni=,no=,inner xsep=,inner ysep=,reverse]                {id}{included}{label}
\farmI    [ni=,no=,f1=,inner xsep=,inner ysep=,reverse]            {id}{included}{label}
\farmII   [ni=,no=,f1=,f2=,inner xsep=,inner ysep=,reverse]        {id}{included}{label}
\farmIII  [ni=,no=,f1=,f2=,f3=,inner xsep=,inner ysep=,reverse]    {id}{included}{label}
\farmIV   [ni=,no=,f1=,f2=,f3=,f4=,inner xsep=,inner ysep=,reverse]{id}{included}{label}
\pipe     [ni=,no=,inner xsep=,inner ysep=,reverse]                {id}{included}{label}
\pipeI    [ni=,no=,f1=,inner xsep=,inner ysep=,reverse]            {id}{included}{label}
\pipeII   [ni=,no=,f1=,f2=,inner xsep=,inner ysep=,reverse]        {id}{included}{label}
\pipeIII  [ni=,no=,f1=,f2=,f3=,inner xsep=,inner ysep=,reverse]    {id}{included}{label}
\pipeIV   [ni=,no=,f1=,f2=,f3=,f4=,inner xsep=,inner ysep=,reverse]{id}{included}{label}
\reduce   [ni=,no=,inner xsep=,inner ysep=,reverse]                {id}{included}{label}
\reduceI  [ni=,no=,f1=,inner xsep=,inner ysep=,reverse]            {id}{included}{label}
\reduceII [ni=,no=,f1=,f2=,inner xsep=,inner ysep=,reverse]        {id}{included}{label}
\reduceIII[ni=,no=,f1=,f2=,f3=,inner xsep=,inner ysep=,reverse]    {id}{included}{label}
\reduceIV [ni=,no=,f1=,f2=,f3=,f4=,inner xsep=,inner ysep=,reverse]{id}{included}{label}

\unzipx     [reverse]         {id}{position}
\zipx       [reverse]         {id}{position}
\unzipv     [reverse]         {id}{position}
\zipv       [reverse]         {id}{position}
\splitatv   [f1=,reverse]     {id}{position}
\catv       [reverse]         {id}{position}
\oddsv      [reverse]         {id}{position}
\evensv     [reverse]         {id}{position}
\reversev   [reverse]         {id}{position}
\groupv     [reverse]         {id}{position}
\concatv    [reverse]         {id}{position}
\filteridxv [f1=,reverse]     {id}{position}
\gatherv    [f1=,f2=,reverse] {id}{position}
\gatherAdpv [f1=,f2=,reverse] {id}{position}
\selectv    [reverse]         {id}{position}
\distributev[f1=,reverse]     {id}{position}
\filterv    [f1=,reverse]     {id}{position}
\getv       [f1=,reverse]     {id}{position}

\visualoddsv   [reverse]{id}{pos}
\visualevensv  [reverse]{id}{pos}
\visualreversev[reverse]{id}{pos}
\visualgroupv  [reverse]{id}{pos}
\visualconcatv [reverse]{id}{pos}
\end{lstlisting}


%%% Local Variables:
%%% TeX-command-default: "Make"
%%% mode: latex
%%% TeX-master: "../refman"
%%% End:



\end{document}
%%% Local Variables:
%%% TeX-command-default: "Make"
%%% mode: latex
%%% TeX-master: t
%%% End:
