%%%%%%%%%%%%%%%
% ENVIRONMENT %
%%%%%%%%%%%%%%%

\newif\ifnolabel
\newif\ifnocolor
\pgfkeys{
  /tikz/no moc color/.is if=nocolor,
  /tikz/no moc label/.is if=nolabel,
  /tikz/type style/.store in = \typeStyle,
  /tikz/type style = \scriptsize\texttt,
  /tikz/label style/.store in = \labelStyle,
  /tikz/label style = \textbf,
  /tikz/function style/.store in = \funcStyle,
  /tikz/function style = \scriptsize,
  /tikz/token pos/.store in = \tokPos,
  /tikz/token pos = 0.5,
  /tikz/deviate/.store in = \pDeviate,
  /tikz/deviate = 0pt,
  /tikz/alwaystrue/.is toggle,
  /tikz/alwaystrue = true,
  /tikz/as/.store in = \pIntName,
  /tikz/as = int,
}

%%%%%%%% 
% KEYS %
%%%%%%%%
\pgfkeys{/forsyde keys/.is family, /forsyde keys,
  %shape & color  
  hasmoc/.is toggle,
  class/.estore in = \fsdClass,
  type/.estore in  = \fsdType,
  shape/.estore in = \fsdShape,
  moc/.style       = {hasmoc, class=#1},
  % separation
  inner xsep/.estore in = \fsdInnerXSep,
  inner ysep/.estore in = \fsdInnerYSep,
  inner sep/.style      = {inner xsep = #1, inner ysep = #1 },
  % position
  anchor/.estore in = \fsdAnchor,
  xshift/.estore in = \fsdXShift,
  yshift/.estore in = \fsdYShift,
  at/.estore in     = \fsdAtInit,
  left of/.style  = {at={#1.west}, xshift=1cm, anchor=east},
  right of/.style = {at={#1.east}, xshift=-1cm, anchor=west},
  above of/.style = {at={#1.north}, yshift=1cm, anchor=south},
  below of/.style = {at={#1.south}, yshift=-1cm, anchor=north},
  % rotation
  rotate shape/.estore in = \fsdRotateShape,
  rotate/.estore in       = \fsdRotate,
  % ports & functions
  npw/.estore in = \fsdNPortLeft,
  npe/.estore in = \fsdNPortRight,
  nf/.estore in  = \fsdNFunctions,
  f1/.estore in  = \fsdFunA,
  f2/.estore in  = \fsdFunB,
  f3/.estore in  = \fsdFunC,
  f4/.estore in  = \fsdFunD,
  ni/.style      = {npw = #1 },
  no/.style      = {npe = #1 },
  f/.code        = {\makeFunctions{#1}},
  f/.default=$f_1$,
  % default values
  default/.style = {
    hasmoc        = false,
    moc           = none,
    type          =,
    shape         = rectangle,
    anchor        = center,
    inner ysep    = 3pt,
    inner xsep    = 5pt,
    rotate        = 0,
    rotate shape  = 0,
    xshift        = 0pt,
    yshift        = 0pt,
    at            = {0,0},
    npw=1, npe=1, nf=0, 
    f1=$ f_1 $, f2=$ f_2 $, f3=$ f_3 $, f4=$ f_4 $,
  },
  %%%%%%%%%%%%%%%%%%%%%%%%%%%
  % SHAPES OF MAIN ELEMENTS %
  %%%%%%%%%%%%%%%%%%%%%%%%%%%
  primitive/.style ={%
    shape=atom shape
    },
  primitiven/.style={%
    shape=nary atom shape
    },
  process/.style={%
    hasmoc,
    shape=leaf shape,
    },
  processn/.style={% 
    hasmoc,
    shape=leafn shape,
    },
  composite/.style={%
    shape=comp shape
    },
  compositen/.style={%
    shape=nary comp shape,
    },
  embed/.style={%
    hasmoc,
    shape=leaf shape,
    inner sep=15pt,
    },
  farmstyle/.style={%
    shape = dp shape,
    inner xsep = 15pt,
    inner ysep = 20pt,
    },
  pipestyle/.style={%
    shape = pipe shape,
    inner xsep = 15pt,
    inner ysep = 20pt,
    },
  skeleton/.style={%
    shape = generic skel shape,
    inner xsep = 15pt,
    inner ysep = 20pt,
    },
  transition/.code = {%
    \edef\theshape{trans shape #1}
    \tikzset{/forsyde keys/shape = \theshape}
    },
  transition/.default={v1}{v1},
  zipx/.style = {%
    transition={s1}{v1},
    rotate shape=180,
    type=zipx,
    },
  unzipx/.style = {%
    transition={s1}{v1},
    type=unzipx,
    },
}

\ExplSyntaxOn

\NewDocumentCommand{\makeFunctions}{m} {%
  \yourb_count_char:nn { ; } { #1 }  
  \pgfmathtruncatemacro{\nFunc}{\l_yourb_count_char_int + 1};
  \def\nnnnFunc{\nFunc}
  \tikzset{/forsyde\space keys/nf = \nFunc}
  \pgfmathtruncatemacro{\countFun}{0};
  \setFunctions{ #1 }
}
\int_new:N \l_yourb_count_char_int
\cs_new_protected:Npn \yourb_count_char:nn #1 #2 {
  \regex_count:nnN { #1 } { #2 } \l_yourb_count_char_int
}
\NewDocumentCommand{\setFunctions}{ >{ \SplitList { ; } } m } {
  \ProcessList { #1 } { \davs__tokens_setfun_rec:n }
}   
\cs_new_protected:Nn \davs__tokens_setfun_rec:n {
  \def\currentfunckey{/forsyde\space keys/f\countFun}
  \pgfmathtruncatemacro{\countFun}{\countFun + 1}
  \tikzset{\currentfunckey = {#1}}
}


\NewDocumentCommand{\listKeys}{ >{ \SplitList { , } } m } {
  \ProcessList { #1 } { /tikz/#1, }
}
\ExplSyntaxOff

\newcommand\addcase[3]{%
  \expandafter\def\csname\string#1@case@#2\endcsname{#3}%
}
\newcommand\makeswitch[2][]{%
  \newcommand#2[1]{%
    \ifcsname\string#2@case@##1\endcsname\csname\string#2@case@##1\endcsname\else#1\fi%
  }%
}

\endinput
