
\section{Getting {{\sc ForSyDe-Atom}}}
\label{sec:getting-forsyde-atom}

The \textsc{ForSyDe-Atom} EDSL can be downloaded from \url{https://github.com/forsyde/forsyde-atom}. The main page contains enough information for acquiring the dependencies and installing the library on your own. However, each example project from the \href{https://github.com/forsyde/forsyde-atom-examples}{\texttt{forsyde-atom-examples}} repository comes with a set of installation scripts written for user convenience, which should be enough for traversing this manual.

\begin{mdframed}[style=attention,frametitle=Attention!]
Be advised that different chapters of the book have been written during different development stages of \textsc{ForSyDe-Atom} and are compatible with different library releases. The code listed throughout the book \emph{might} be incompatible with the latest release. We strongly recommend using the installation scripts included in each project associated with a chapter, which acquire and install the right dependencies in a local sandbox.
\end{mdframed}

Provided you have an OS installation where the minimum dependencies (\href{https://www.gnu.org/software/make/}{GNU make}, a \href{https://git-scm.com/downloads}{Git CLI client}, \href{https://www.haskell.org/platform/}{Haskell Platform} and \href{https://www.haskell.org/cabal/download.html}{cabal-install}) are working and accessible by your user profile, to run the \texttt{getting-started} example associated with \cref{ch:getting-started}, you simply need to type in the terminal:

\begin{verbatim}
# download the examples
git clone https://github.com/forsyde/forsyde-atom-examples.git

# change directory to the desired project folder
cd forsyde-atom-examples/getting-started

# install the project (and dependencies) in a sandbox
make install

# open an interpreter session with the examples loaded
cabal repl
\end{verbatim}

%%% Local Variables:
%%% mode: latex
%%% TeX-master: "manual"
%%% End:
