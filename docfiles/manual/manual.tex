\documentclass{book}
\usepackage{../atom-manual}
\usepackage{../urls}

\title{\textsc{ForSyDe-Atom}\\User Manual}
\author{George Ungureanu}

\newcommand*{\RootPath}{../../}%


\addbibresource{../refs.bib}

\begin{document}



\pagenumbering{Roman} 
\dominitoc

\includepdf[pages={1}]{title.pdf}

\tableofcontents
\clearpage
\listoffigures
\mainmatter
\pagenumbering{arabic}

\begin{refsection}
\chapter{Introduction}
\label{cha:introduction}

\begin{summary}
  In this chapter we introduce the purpose, organization and usage of this document, as well as brief instructions and references for helping to set up the \textsc{ForSyDe-Atom} libraries. The scope is to facilitate the reader's progression through this document.
\end{summary}

\section{Purpose \& organization}
\label{sec:purp-organ}

This book is a living document which gathers material related to \textsc{ForSyDe-Atom} and binds it in form of a user manual. The vast majority of text contained by this book originates from actual inline or literate source code documentation, in form of examples, tutorials, reports and even library API documentation. This means that this document evolves with the \textsc{ForSyDe-Atom} project itself and is periodically updated.

\textsc{ForSyDe-Atom} is a shallow-embedded DSL in the functional programming language Haskell for modeling cyber-physical and parallel systems. It enforces a disciplined way of modeling by separating the manifold concerns of systems into orthogonal \emph{layers}. The \textsc{ForSyDe-Atom} formal framework aims to providing (where possible) a minimum set of primary common building blocks for each layer called \emph{atoms}, capturing elementary semantics. Even so, the modeling framework provides library blocks and modules commonly used in CPS defined in terms of \emph{patterns} of atoms. For more information on \textsc{ForSyDe-Atom} itself, please consult the associated scientific publications or the API extended documentation\footnote{currently available online at \url{https://forsyde.github.io/forsyde-atom/}}.

This document is structured in two parts. The first part gathers examples, tutorials and experiments in a learning progression. Each chapter associated with (and actually generated from) a \href{https://www.haskell.org/cabal/}{Haskell Cabal project} included in the \texttt{forsyde-atom-examples}\footnote{available online at \url{https://github.com/forsyde/forsyde-atom-examples}} repository. This means that the first part of the book is meant to be read in parallel with running the associated example project which conveniently exports functions to test the listed code ``on-the-fly''. The second part of the book is the actual inline API documentation of \textsc{ForSyDe-Atom} generated with \href{https://www.haskell.org/haddock/}{Haddock}, which serves also as an extended library report and provides information both on theoretical and implementation issues.

\begin{summary}
  If for any reason you have difficulties following the document or you encounter bugs or discrepancies in the code and text do not hesitate to contact the author(s) or maintainer(s) on GitHub or by email. 
\end{summary}

\section{Getting {{\sc ForSyDe-Atom}}}
\label{sec:getting-forsyde-atom}

The \textsc{ForSyDe-Atom} EDSL can be downloaded from \url{https://github.com/forsyde/forsyde-atom}. The main page contains enough information for acquiring the dependencies and installing the library on your own. However, each example project from the \href{https://github.com/forsyde/forsyde-atom-examples}{\texttt{forsyde-atom-examples}} repository comes with a set of installation scripts written for user convenience, which should be enough for traversing this manual.

Provided you have an OS installation where the minimum dependencies (\href{https://www.gnu.org/software/make/}{GNU make}, a \href{https://git-scm.com/downloads}{Git CLI client}, \href{https://www.haskell.org/platform/}{Haskell Platform} and \href{https://www.haskell.org/cabal/download.html}{cabal-install}) are working and accessible by your user profile, to run the \texttt{getting-started} example associated with \cref{ch:getting-started}, you simply need to type in the terminal:

\begin{verbatim}
# download the examples
git clone https://github.com/forsyde/forsyde-atom-examples.git

# change directory to the desired project folder
cd forsyde-atom-examples/getting-started

# install the project (and dependencies) in a sandbox
make install

# open an interpreter session with the examples loaded
cabal repl
\end{verbatim}

\section{Using this document}
\label{sec:using-this-document}

\textbf{DISCLAIMER:} the document assumes that the reader is familiar with the syntax of Haskell and the usage of a Haskell interpreter (e.g. \texttt{ghci}). Otherwise, we recommend consulting at least the introductory chapters of one of the following books by \cite{Lipovaca11} and \cite{Hutton07}.

This document has been created using literate programming. This means that all code shown in the listings is compilable and executable. There are two types of code listing found in this document. This style
\begin{code}
-- | API documentation comment 
myIdFunc :: a -> a
myIdFunc = id
\end{code}
shows \emph{source code} as it is found in the implementation files. Notice that in-line API documentation is also shown as comments. This style
\begin{interactive}
Prelude> 1 + 1
2
\end{interactive}
suggests \emph{interactive commands} given by the user in a terminal or an interpreter session. The listing above shows a typical \texttt{ghci} session, where the string after the prompter symbol \texttt{>} suggests the user input (e.g. \texttt{1 + 1}). Whenever relevant, the expected output is printed one row below that (e.g. \texttt{2}).

The code examples are bundled as separate \href{https://www.haskell.org/cabal/}{Cabal} packages and is provided as libraries meant to be loaded in an interpreter session in parallel with reading this document. Detailed instructions on how to install the packages can be found in the \texttt{README.md} file in each project. The best way to install the packages is within sandboxed environments with all dependencies taken care of, usually scripted within the \texttt{make} commands. After a successful installation, to open an interpreter session pre-loaded with the main sandboxed library, you just need to type in the following command in a terminal from the package root path (the one containing the \texttt{.cabal} file):
\begin{interactive}
# cabal repl
\end{interactive}

Each section of this document contains a small example written within a library \emph{module}, like:
\begin{code}
module X where
\end{code}
One can access all functions in module \texttt{X} by importing it in the interpreter session, unless otherwise noted (e.g. library \texttt{X} is re-exported by \texttt{Y}).
\begin{interactive}
*Y> import X
\end{interactive}
Now suppose that function \texttt{myIdFunc} above was defined in module \texttt{X}, then one would have direct access to it, e.g.:
\begin{interactive}
*Y X> :t myIdFunc
myIdFunc :: a -> a
*Y X> myIdFunc 3
3  
\end{interactive}
By all means, the code for \texttt{myIdFunc} or any source code for that matter can be copied/pasted in a custom \texttt{.hs} file and compiled or used in any relevant means. The current format was chosen because it is convenient to ``get your hands dirty'' quickly without thinking of issues associated with compiler suites.

A final tip: if you think that the full name of \texttt{X} is polluting your prompter or is hard to use, then you can import it using an alias:
\begin{interactive}
*Y> import Extremely.Long.Full.Name.For.X as ShortAlias
*Y ShortAlias> 
\end{interactive}

\printbibliography[heading=subbibliography]
\end{refsection}

\part{Examples \& Reports}
\label{part:introduction}

\begin{refsection}
\chapter{Getting Started with {{\sc ForSyDe-Atom}}}
\label{ch:getting-started}
\renewcommand*{\RootPath}{../../getting-started/docs/latex}%
\graphicspath{{../../getting-started/docs/latex/figs/}}

\begin{summary}
  In this chapter we introduce the purpose, organization and usage of this document, as well as brief instructions and references for helping to set up the \textsc{ForSyDe-Atom} libraries. The scope is to facilitate the reader's progression through this document.

%%% Local Variables:
%%% mode: latex
%%% TeX-master: "manual"
%%% End:

\end{summary}
\minitoc


\section{Goals}
\label{sec:hybrid:goals}

The reader is assumed to have been familiarized with the \textsc{ForSyDe-Atom} modeling framework. A good resource for that is \chapterref{ch:getting-started}. The main goals of this \SelfRef\ are:
\begin{itemize}
\item explore alternative \textsc{ForSyDe-Atom} models for widely known hybrid (discrete and continuous time) systems, with the scope of analyzing the implications from the modeling and simulation perspective.
\item train the reader into the decision-making process of modeling hybrid systems, and the trade-offs involved. While as per writing this report the \textsc{ForSyDe-Atom} modeling framework is still limited, future directions are hinted.
\item support the associated scientific publications with the complete experiments and results for the models used as case studies. These publications include: \cite{ungureanu18a}.
\end{itemize}


%%% Local Variables:
%%% TeX-command-default: "Make"
%%% mode: latex
%%% TeX-master: "../report"
%%% End:


\section{The basics}
\label{sec:basic-usage}
\input{\RootPath/code/GettingStarted/Basics}

\subsection{Visualizing your data}
\label{sec:visu-your-data}
\input{\RootPath/code/GettingStarted/Plot}

\section{Toy example: a focus on MoCs}
\label{sec:toy-example}
This example has been used as a case study for introducing the new concepts of \textsc{ForSyDe-Atom} in the paper of \cite{ungureanu17}. It describes the simple system from \cref{fig:subfig1} which exposes four layers, structured like in \cref{fig:toy-4layer}. This system is then fed vectors of signals describing different MoCs and its response is observed. In \cref{fig:subfig2,fig:subfig3,fig:subfig4,fig:subfig5} some possible projections on the different layers are depicted. For now they are used just as trivia, and you need not bother with them that much.
\vspace{-.4cm}
\begin{figure}[ht!]
  \centering
  \subfloat[The four layered structure of the toy example]{
    \includegraphics[width=.45\textwidth]{toy-4layer}
    \label{fig:toy-4layer}}
  \hfill
  \subfloat[View from skeleton layer]{
    \includegraphics[width=0.5\textwidth]{view_example_skeleton.pdf}
    \label{fig:subfig1}}%
  \hfill
  \subfloat[Top view from process layer]{
    \includegraphics[width=0.47\textwidth]{view_example_unraveled.pdf}
    \label{fig:subfig2}}
  \hfill
  \subfloat[Flattened \& refined view from process layer]{
    \includegraphics[width=0.47\textwidth]{view_example_flat.pdf}
    \label{fig:subfig3}}
  \hfill
  \subfloat[View from behavior layer, as projected by a timed MoC]{
    \includegraphics[width=0.47\textwidth]{view_example_behavior.pdf}
    \label{fig:subfig4}}
  \hfill
  \subfloat[View from function layer, as projected by a timed MoC]{
    \includegraphics[width=0.47\textwidth]{view_example_function.pdf}
    \label{fig:subfig5}}
\caption{Views and projections for the toy system}\label{fig:application}
\end{figure}

This is a synthetic example meant to introduce as many concepts as possible in a short amount of time and, among others, it highlights:
\begin{itemize}
\item the power of partial application for creating parameterized structures, such as the process network for \texttt{stage1}.
\item alternative designs for the same \texttt{toy} system to show the effect of MoCs. First it is instantiated using different process constructor helpers defined for each MoC separately. Afterwards it is written as one single polymorphic instance using MoC layer patterns, overloaded with execution semantics in accordance with the tag system injected into the system.
\end{itemize}

For the sake of brevity, we also provide the functional description in the language introduced by \cite{ungureanu17} in eqs.~\eqref{eq:case1}--\eqref{eq:moore} and \cref{tab:application}. Do not bother much about this notation either, as this exact definition will appear in the code in a more ``human readable'' form.

\begin{align}
  \mathtt{toy}& : \SkelVec{V} \rightarrow
                \SkelVec{S} \rightarrow S \label{eq:case1}\\
  \mathtt{toy} &\SkelVec{i}\SkelVec{s} =
                 \mathtt{when}_{\textsc{m}}(\context{\Gamma_w}{w_{\textsc{b}}}) \circ
                 \mathtt{reduce}_{\textsc{s}}(r_{\textsc{m}}) \circ
                 \mathtt{map}_{\textsc{s}}(pc_{\textsc{m}})\SkelVec{i}\SkelVec{s}
                 \nonumber
\end{align}%
% \hspace{.2cm}
where%
\begin{align}
  \mathtt{when}_{\textsc{m}} (\context{\Gamma_w}{w_{\textsc{b}}})(s)
  & = ((\bot\ \BhPhi) \circ (\context{\Gamma_w}{w_{\textsc{b}}}))
    \oplus s \label{eq:when}\\
  r_{\textsc{m}}(x,y) &= \context{\Gamma_r}{r_{\textsc{b}}} \oplus (x,y) \\
  \mathtt{map_{s}}(pc_{\textsc{m}})\SkelVec{v}\SkelVec{s}
  &= pc_{\textsc{m}} \SkelFun \SkelVec{v} \SkelApp
    \SkelVec{s} \label{eq:re-map}\\
  pc_{\textsc{m}}(x,y)
  &= \mathtt{moore}_{\textsc{m}} (%
    \context{\Gamma_{\mathit ns}}{ns_{\textsc{b}}},
    \context{\Gamma_{\mathit od}}{od_{\textsc{b}}},x)(y)\label{eq:moore}
\end{align}

\def\vecbegin#1{\multicolumn{1}{@{ $\langle$}c}{#1}}%
\begin{table}[ht!]\scriptsize\setlength{\tabcolsep}{5pt}\centering
  \caption{\sc Contexts, Functions and Initial Tokens for
    the System in Eq. \eqref{eq:case1}}\label{tab:application}
  \hspace*{-3cm}%
  {\renewcommand{\arraystretch}{1.5}%
    \begin{tabular}{c|c@{$\vdash$}c|c@{ $\vdash$}c|c@{$\vdash$}c|c@{$\vdash$}c|c@{ }c@{ }c@{ }c@{$\rangle$}}
      \toprule
      MoC & $\Gamma_{\mathit w}$ & $w_{\textsc{b}}(x)$ & $\Gamma_{\mathit r}$ & $r_{\textsc{b}}(x,y)$ & $\Gamma_{\mathit ns}$ & $ns_{\textsc{b}}(x,y)$ & $\Gamma_{\mathit od}$ & $od_{\textsc{b}}(x)$ & \multicolumn{4}{c}{$\SkelVec{i}=\SkelVec{(t,v)}$} \\    
      \midrule
      SDF\footnote{$\Gamma_{\text{SDF}} = (\text{consumption rate for first input}[,\text{consumption rate for second input}]), \text{production rate}$}
      & 2,2 & $(x_1<0,x_2<0) \BhDef$ & (1,1),1 & $(x_1+y_1) \BhDef$ & (1,2),1 & $(x_1+y_1+y_2) \BhDef$ & 1,1 & $x_1 \BhDef$ 
      & \vecbegin{$(\ ,-1)$} & $(\quad,1)$ & $(\quad,-1)$ & $(\quad,1)$ \\
      SY 
      &     & $(x<0) \BhDef$        &          & $(x+y) \BhDef$    &         & $(x+y) \BhDef$        &     & $x\ \BhDef$ 
      & \vecbegin{$(\ ,-1)$} & $(\quad,1)$ & $(\quad,-1)$ & $(\quad,1)$ \\
      DE 
      &     & $(x<0) \BhDef$        &          & $(x+y) \BhDef$    &         & $(x+y) \BhDef$        &     & $x\ \BhDef$ 
      & \vecbegin{$(0.5,-1)$} & $(1.4,1)$ & $(1.0,-1)$ & $(1.4,1)$ \\
      CT 
      &     & $(x<0) \BhDef$        &          & $(x+y) \BhDef$    &         & $(x+y) \BhDef$        &     & $x\ \BhDef$ 
      & \vecbegin{$(1, \lambda t\rightarrow -1)$} & $(1.4,\lambda t\rightarrow 1)$ & $(1,\lambda t\rightarrow -1)$ & $(1.4,\lambda t\rightarrow 1)$ \\
      \bottomrule
    \end{tabular}
}\end{table}


%%% Local Variables:
%%% TeX-command-default: "Make"
%%% mode: latex
%%% TeX-master: "../report"
%%% End:


\subsection{Test input signals}
\label{sec:test-signals}
\input{\RootPath/code/GettingStarted/TestSignals}

\subsection{SY instance}
\label{sec:sy-instance}
\input{\RootPath/code/GettingStarted/SY}

\subsection{DE instance}
\label{sec:de-instance}
\input{\RootPath/code/GettingStarted/DE}

\subsection{CT instance}
\label{sec:ct-instance}
\input{\RootPath/code/GettingStarted/CT}

\subsection{SDF instance}
\label{sec:sdf-instance}
\input{\RootPath/code/GettingStarted/SDF}

\subsection{Polymorphic instance}
\label{sec:poly-instance}
\input{\RootPath/code/GettingStarted/Polymorphic}

\section{Making your own patterns}
\label{sec:making-your-own}
\input{\RootPath/code/GettingStarted/CustomPattern}

\label{sec:conclusion}
\textit{This project is still under development...}

%%% Local Variables:
%%% mode: latex
%%% TeX-master: "../report"
%%% End:


\printbibliography[heading=subbibliography]
\end{refsection}

\begin{refsection}
\chapter{Hybrid CT/DT Models in {{\sc ForSyDe-Atom}}}
\label{ch:hybrid}
\renewcommand*{\RootPath}{../../hybrid/docs/latex}%
\graphicspath{{../../hybrid/docs/latex/figs/}}


\begin{summary}
In this chapter we introduce the purpose, organization and usage of this document, as well as brief instructions and references for helping to set up the \textsc{ForSyDe-Atom} libraries. The scope is to facilitate the reader's progression through this document.

%%% Local Variables:
%%% mode: latex
%%% TeX-master: "manual"
%%% End:

\end{summary}
\minitoc


\section{Goals}
\label{sec:hybrid:goals}

The reader is assumed to have been familiarized with the \textsc{ForSyDe-Atom} modeling framework. A good resource for that is \chapterref{ch:getting-started}. The main goals of this \SelfRef\ are:
\begin{itemize}
\item explore alternative \textsc{ForSyDe-Atom} models for widely known hybrid (discrete and continuous time) systems, with the scope of analyzing the implications from the modeling and simulation perspective.
\item train the reader into the decision-making process of modeling hybrid systems, and the trade-offs involved. While as per writing this report the \textsc{ForSyDe-Atom} modeling framework is still limited, future directions are hinted.
\item support the associated scientific publications with the complete experiments and results for the models used as case studies. These publications include: \cite{ungureanu18a}.
\end{itemize}


%%% Local Variables:
%%% TeX-command-default: "Make"
%%% mode: latex
%%% TeX-master: "../report"
%%% End:


\section{RC Oscillator}
\label{sec:rc-oscillator}
\input{\RootPath/code/Hybrid/RCOsc}

\section{Conclusion}
\label{sec:conclusion}
\textit{This project is still under development...}

%%% Local Variables:
%%% mode: latex
%%% TeX-master: "../report"
%%% End:


\printbibliography[heading=subbibliography]
\end{refsection}

\part{API Documentation}
\label{part:api-documentation}

\begin{summary}
  This section is currently under construction. For the moment, please refer to online API documentation at \url{https://forsyde.github.io/forsyde-atom/}
\end{summary}

\end{document}

%%% Local Variables:
%%% mode: latex
%%% TeX-master: t
%%% End:
